\documentclass[a4paper,11pt]{report}
%\documentclass{foils}

\usepackage{geometry,graphicx}
\usepackage{verbatim}
\usepackage{url}
\usepackage{fancybox}
\usepackage{fancyhdr}
\usepackage{framed}
\usepackage{color}
\usepackage{multirow}

\usepackage{lingmacros}  % included with GenI
\usepackage{covington}   % included with GenI

% ----------------------------------------
\newcommand{\jargon}{\textbf}
\newcommand{\natlang}{\textit}
\newcommand{\semexpr}{\texttt}
\newcommand{\tautree}[1] {$\tau_{#1}$}
\newcommand{\koweytree}{\texttt}
\long\def\ignore#1{}

\newcommand{\fnparam}{\texttt}
\newcommand{\fnref}[1]{\textit{#1} (page \pageref{fn:#1})}
\newcommand{\fnreflite}[1]{\textit{#1}}
\newcommand{\fnlabel}[1]{\paragraph{#1}\label{fn:#1}}

\newcommand{\tuple}[1]{\langle #1 \rangle}
\setcounter{chapter}{-1}
\setcounter{secnumdepth}{3}
% ----------------------------------------

\geometry{verbose,tmargin=40mm,bmargin=40mm,lmargin=25mm,rmargin=25mm}
%\input{lambdaTeX}

\pagestyle{fancyplain} 
\lfoot{Geni source code}
\cfoot{\thepage}
\rfoot{}

%\setlength\parindent{0pt}
\setlength{\fboxsep}{0.1pt}

\renewcommand\FrameHeightAdjust{1pt}

\newenvironment{code}{\comment}{\endcomment}

\begin{document}
\title{Geni manual}
\author{Langue et Dialogue\\LORIA}

\maketitle
\tableofcontents

% -------------------------------------------------------------------------
% Overview
% -------------------------------------------------------------------------

\chapter*{License}

GenI surface realiser\\
Copyright \copyright 2005/2006 Carlos Areces and Eric Kow

\bigskip

This program is free software; you can redistribute it and/or
modify it under the terms of the GNU General Public License
as published by the Free Software Foundation; either version 2
of the License, or (at your option) any later version.

\bigskip

This program is distributed in the hope that it will be useful,
but WITHOUT ANY WARRANTY; without even the implied warranty of
MERCHANTABILITY or FITNESS FOR A PARTICULAR PURPOSE.  See the
GNU General Public License for more details.

\bigskip

You should have received a copy of the GNU General Public License
along with this program; if not, write to the Free Software
Foundation, Inc., 59 Temple Place - Suite 330, Boston, MA  02111-1307, USA.

\chapter{Overview}

This document doesn't officially exist yet.  It's just Eric
experimenting with using Literate Haskell to build the manual
along with the source code \`a la Darcs.

Some of the text here might be pretty nonsensical in a manual.  It's
because in the early days of Literate GenI, I was mostly using literate
programming for writing comments.  Later on, (by reading the darcs
source code),  I discovered that another good use of literate
programming is for embedding your user manual right into your code.
This makes it easier to keep your documentation up to date.  My hope
is that either I or the person who looks after GenI when I'm gone
will complete the transition of the user manual stuff into this style.

Note that in addition to the manual, there is \textit{Literate GenI}
\cite{literateGeni} which actually tries to explain how GenI is
implemented.

% -------------------------------------------------------------------------
% Overview
% -------------------------------------------------------------------------

\part{Manual}

\input{../src/NLP/GenI/Configuration.lhs}
\input{../src/NLP/GenI/GeniParsers.lhs}
\input{../src/NLP/GenI/Gui.lhs}

\part{Optimisations}

\input{../src/NLP/GenI/Polarity.lhs}

{
\bibliographystyle{alpha}
\bibliography{genidoc}
}

\end{document}

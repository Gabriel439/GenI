\documentclass[a4paper,11pt]{report}

\usepackage{geometry,graphicx}
\usepackage{verbatim}
\usepackage{url}
\usepackage{fancybox}
\usepackage{fancyhdr}
\usepackage{fancyvrb}
\usepackage{framed}
\usepackage{color}
\usepackage{multirow}
\usepackage{xspace}

\usepackage{lingmacros}  % included with GenI
\usepackage{covington}   % included with GenI

% ----------------------------------------
\newcommand{\jargon}{\textbf}
\newcommand{\geni}{\textsc{GenI}\xspace}
\newcommand{\natlang}{\textit}
\newcommand{\semexpr}{\texttt}
\newcommand{\tautree}[1] {$\tau_{#1}$}
\newcommand{\koweytree}{\texttt}
\long\def\ignore#1{}

\newcommand{\fnparam}{\texttt}
\newcommand{\fnref}[1]{\textit{#1} (page \pageref{fn:#1})}
\newcommand{\fnreflite}[1]{\textit{#1}}
\newcommand{\fnlabel}[1]{\paragraph{#1}\label{fn:#1}}

\newcommand{\tuple}[1]{\langle #1 \rangle}
\setcounter{chapter}{-1}
\setcounter{secnumdepth}{3}
% ----------------------------------------

\geometry{verbose,tmargin=40mm,bmargin=40mm,lmargin=25mm,rmargin=25mm}

\pagestyle{fancyplain} 
\lfoot{GenI reference manual}
\cfoot{\thepage}
\rfoot{}

\newenvironment{code}{\comment}{\endcomment}
\newenvironment{includecodeinmanual}
{\renewenvironment{code}
  {\VerbatimEnvironment
   \footnotesize
   %\begin{framedcode}
   \begin{Verbatim}
  }
  {\end{Verbatim}
   %\end{framedcode}
   \normalsize }
}
{
  \renewenvironment{code}{\comment}{\endcomment}
}

\begin{document}
\title{GenI reference manual}
\author{TALARIS\\INRIA}

\maketitle
\tableofcontents

% -------------------------------------------------------------------------
% Overview
% -------------------------------------------------------------------------

\chapter{Overview}

This is the GenI reference manual.  We provide four distinct kinds of
documentation.  At the time of this writing, 2009-09-25 all four of them
are woefully incomplete.  In order of most user-oriented to most
developer-oriented the documentation comes in the form of

\begin{enumerate}
\item A ``getting started'' page on\\
      \url{http://trac.haskell.org/GenI/wiki/Walkthrough}
\item Reference manual (this document)
\item \textit{Literate GenI} which provides details on GenI
      algorithms and implementation (see \cite{literateGeni}),
      still not very literate at the time of this writing
      2009-09-24.
\item API: do \verb!cabal haddock!
\end{enumerate}

In addition to the formal documentation, we also provide an FAQ and
other informal notes on the wiki.  See
\url{http://trac.haskell.org/GenI} for details.

\part{Basics}

\input{../src/NLP/GenI/Configuration.lhs}
\input{../src/NLP/GenI/GeniParsers.lhs}
\input{../src/NLP/GenI/Gui.lhs}

\part{Extras}

\input{../src/NLP/GenI/OptimalityTheory.lhs}
\input{../src/NLP/GenI/Morphology.lhs}

{
\bibliographystyle{alpha}
\bibliography{genidoc}
}

\end{document}

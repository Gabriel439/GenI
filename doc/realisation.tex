%\def\semrow#1#2{\multicolumn{#1}{c}{\small \koweysemexpr{#2}}\\}
% ----------------------------------------- Kelvin / Colorado
\def\pntree#1#2#3{%
\begin{tabular}{cc}
\multicolumn{1}{c}{\node{n}{NP$_#3$}} \\[3ex]
\node{nx}{#1} \\[2ex]
\semrow{1}{#2} \\
\nodeconnect{n}{nx}
\end{tabular}%
}

% ----------------------------------------------------------------------
% kelvin runs often
% ----------------------------------------------------------------------

\def\KelvinRunsOftenTrees{%
% -----------------------------------------
\pntree{Kelvin}{l1:name(k Kelvin)}{k}
% -----------------------------------------
&
% ----------------------------------------- runs
\begin{tabular}{cc}
\multicolumn{2}{c}{\hspace{-1em}\node{n}{S}$_{?E2}$}
\\[3ex]
\node{n1}{NP$\downarrow$$^k$}  &
\node{n2}{VP$_{?E1}^{?E2}$}
\\[2ex]
&
\node{n2.1}{V$_r^{?E1}$}
\\[2ex]
&
\node{n2.1x}{runs}
\\[2ex]
\semrow{2}{l0:runs(r) l0:agent(r k)} \\
\nodeconnect{n}{n1} \nodeconnect{n}{n2}
\nodeconnect{n2}{n2.1}
\nodeconnect{n2.1}{n2.1x}
\end{tabular}
% -----------------------------------------
&
% ----------------------------------------- often
\begin{tabular}{cc}
\multicolumn{2}{c}{\hspace{0.5em}\node{n}{VP}$^r$}
\\[3ex]
\node{n1}{VP$_r$*} &
\node{n2}{Adv}
\\[3ex]
&
\node{n2x}{often}
\\[2ex]
\semrow{2}{l0:often(r)}
\nodeconnect{n}{n1} \nodeconnect{n}{n2}
\nodeconnect{n2}{n2x}
\end{tabular}
% -----------------------------------------
}

% ----------------------------------------------------------------------
% in Colorado
% ----------------------------------------------------------------------

\def\InColoradoTrees{%
% ----------------------------------------- in...
\begin{tabular}{ccc}
\multicolumn{3}{c}{\hspace{-1.5em}\node{n}{VP$^r$}}
\\[4ex]
\node{n1}{VP$_r$*} &
\multicolumn{2}{c}{\hspace{-0.5em}\node{n2}{PP}}
\\[3ex]
&
\node{n2.1}{P} &
\hspace{1em}\node{n2.2}{NP$\downarrow$$^i$}
\\[2ex]
&
\node{n2.1x}{in}
\\[2ex]
\semrow{3}{l0:in(r,i)} \\
\nodeconnect{n}{n1} \nodeconnect{n}{n2}
\nodeconnect{n2}{n2.1} \nodeconnect{n2}{n2.2}
\nodeconnect{n2.1}{n2.1x}
\end{tabular}
% -----------------------------------------
&
% -----------------------------------------
\pntree{Colorado}{l3:place(i Colorado)}{i}
% -----------------------------------------
}

% ----------------------------------------------------------------------
% tables
% ----------------------------------------------------------------------

\def\KelvinRunsOften{%
\begin{tabular}{ccc}
&&\\
\KelvinRunsOftenTrees
\end{tabular}
}

\def\KelvinRunsOftenInColorado{%
\begin{tabular}{ccc}
&&\\
\KelvinRunsOftenTrees \\[3ex]
\InColoradoTrees \\
\end{tabular}
}




\def\rulename#1#2{\multicolumn{1}{r#1}{\tiny (#2)}} % for deductive parsing
\newcommand{\ellyou}{L$_U$\xspace}
\newcommand{\koweytautree}[1] {$\tau_{#1}$}
\newcommand{\koweynatlang}[1]{``#1''}
\newcommand{\koweytuple}[1]{\langle #1 \rangle}
\newcommand{\koweysemexpr}{\textit}

\chapter{Realisation Algorithm}
\label{cha:geni}

\textbf{TODO} : This imported from my PhD thesis.  It needs massaging.

\geni uses an FB-LTAG grammar with an \ellyou semantics and a bottom-up
chart generation algorithm.  Realisation takes place in two broad
phases: lexical selection, which returns a set of elementary trees, and
tree assembly, which combines the selected trees in reasonable ways.

\subsection{Lexical selection}
\label{sec:lexical-selection}

Now, on to the two phases of \geni.
For a parser, lexical selection might consist of collecting the
elementary trees that correspond to the words of the input sentence.
This is similar for realisation except that we collect the elementary
trees that correspond to literals of the input semantics, more precisely
those trees whose semantics subsumes part of the input semantics.

The input to lexical selection is thus (i) a flat input semantics and
(ii) an FB-LTAG grammar.  The grammar consists of a set of lexical items
consisting of an elementary tree and
a lexical semantics.  More precisely:
\begin{definition}[lexical item]\label{def:lexical-item}
A lexical item consists of a pair $\koweytuple{T,S}$, where $T$ is
an LTAG elementary tree and
$S$ is an \ellyou formula.  Unification variables in the lexical item
are understood to have scope over the whole item, that is $T$ and $S$.
\end{definition}

In the lexical selection process, we basically retrieve any lexical item
whose semantics is non-empty and can be unified with some part of the
input semantics.  The unification variables in the lexical semantics are
shared with the associated elementary tree.  Via the unification of the
(subsumed part of the) input and lexical semantics, semantic indices
from the input are propagated into the feature  structures of the
elementary tree.  Because the input semantics is saturated, the
instantiated lexical item also has a saturated semantics.

\subsection{Tree assembly (chart generation)}
\label{sec:geni-chart}

After the lexical selection phase, we move into tree assembly.  The
input to this phase is the semantic formula we wish to
realise, and the set of lexically selected elementary trees.  Our
objective is to find every derived tree that can be built from these
parts and which is (i) syntactically complete, meaning it has no empty 
substitution sites and (ii) semantically complete, meaning that its
semantics exactly matches the input semantics.  Ultimately, the goal
of realisation is to produce a set of strings, but this basically
consists of concatenating the leaves of each derived tree.

To cope with the problem of intersective modifiers, the algorithm uses
the delayed modification strategy of \cite{carroll1999ecg}.  Realisation
occurs in two chart generation phases, a substitution phase (where only
TAG substitutions are performed), and an adjunction phase (likewise, TAG
adjunctions only).  Separating forces the realiser to only insert
modifiers into syntactically complete structures.  It also happens to be
a very natural strategy to use with TAG because adjunction and auxiliary
trees are part and parcel of the formalism.  This could be seen as a
potential advantage for TAG as a generation formalism
\cite{gardent2005gas}.

% \begin{table}
% \begin{center}
% \footnotesize
% \begin{tabular}{|p{4em}|c|}
% \hline
% \multicolumn{2}{|l|}{Given an input semantics $\mathbf{sem}$ and lexicon $L$:} \\
% \hline
% Axioms    & \genisubinit \\[5ex]
% Goals     & \genisubgoal \\[3ex]
% Inference rules& \genisubcomp \\
%           & \rulename{|}{Sub}\\[2ex]
% \hline
% \end{tabular}
% \end{center}
% \caption{\label{tbl:genisub-rules}\geni substitution phase (modulo unification)}
% \end{table}
% 
% \begin{table}
% \begin{center}
% \footnotesize
% \begin{tabular}{|p{4em}|c|}
% \hline
% \multicolumn{2}{|l|}{Given an input semantics $\mathbf{sem}$, and
% goal edge items $GS$ from the previous phase}\\
% \hline
% Axioms    & \geniadjinit \\[5ex]
% Goals     & \geniadjgoal \\[3ex]
% Inference rules& \geniadjcomp \\
%           & \rulename{|}{Adj}\\[2ex]
% \hline
% \end{tabular}
% \end{center}
% \caption{\label{tbl:geniadj-rules}\geni adjunction phase (modulo unification)}
% \end{table}

We can think of the two phases as two successive chart generation
problems fed to a generic deductive parser.  For clarity, here is a more
procedural description of the algorithm.  We are using the usual agenda
based control (Section \ref{sec:agenda-based-control}), where the
\jargon{agenda} is a data structure for storing unprocessed intermediate
results.  For convenience, we introduce a new data structure, called an
\jargon{auxiliary agenda}.  The auxiliary agenda is used to set aside
any syntactically complete auxiliary trees found during the first phase
of realisation.  It is not strictly necessary, but it saves the trouble
of filtering the chart afterwards.  Here are the two phases in detail:

\paragraph{Substitution phase}
\begin{enumerate}
\item Store the lexically selected trees in
  the agenda, except for auxiliary trees which are devoid of
  substitution nodes (put these in the auxiliary agenda).\\
\item Loop: Retrieve a tree from the agenda, add it to the chart and try
  to combine it by substitution with trees present in the chart. Add any
  resulting derived tree to the agenda. Stop when the agenda is empty.
\end{enumerate}

\paragraph{Adjunction phase}
\begin{enumerate}
\setcounter{enumi}{2}
\item Move the chart trees to the agenda and the auxiliary agenda trees
      on to the chart.  Discard all trees which are not syntactically
      complete (i.e. that still have open substitution nodes), as they
      will never become complete in this phase.
\item Loop: Retrieve a tree from the agenda, add it to the chart and try to
  combine it by adjunction with trees present in the chart. Add any
  resulting derived tree to the agenda. Stop when the agenda is empty.
\end{enumerate}

\noindent
The inner workings of this algorithm might be better illustrated by an
example or two:

%\subsubsection{Kelvin runs often}
%\begin{figure}[htpb]
%\begin{center}
%\KelvinRunsOften
%\end{center}
%\caption{Kelvin runs often}
%\label{fig:kro}
%\end{figure}

Let us start with a simple one. Suppose that the input semantics is
\koweysemexpr{l0:run(r), l0:agent(r,k), l0:often(r), l1:name(k,Kelvin)}.
Lexical selection gives us a set of elementary
trees lexicalised with the words \koweynatlang{Kelvin},
\koweynatlang{often}, \koweynatlang{runs} (Figure \ref{fig:kro}).  The
trees for \koweynatlang{Kelvin} and \koweynatlang{runs} are placed on
the agenda, the one for \koweynatlang{often} is placed on the auxiliary
agenda.

We begin with the \jargon{substitution phase}, which consists of
systematically exploring the possibility of combining two trees by
substitution.  Note that in the table below, the letters `k', `r' and
`o' stand for the elementary trees that correspond to
\koweynatlang{Kelvin}, \koweynatlang{runs} and \koweynatlang{often},
respectively.  When the trees combine, we write, for example, `kr' to
mean a derived tree for \koweynatlang{Kelvin runs}.  We see that here, the
tree for \koweynatlang{Kelvin} is substituted into the one for
\koweynatlang{runs}, and the resulting derived tree for
\koweynatlang{Kelvin runs} is placed on the agenda.  Trees on the agenda
are processed one by one in this fashion, although in this simple
example, there is only one substitution to be done.  When the agenda is
empty, indicating that all combinations have been tried, we prepare for
the next phase.

\begin{center}
{\footnotesize
\begin{tabular}{|p{6em}|p{9em}|p{9em}|p{4em}|}
\hline
Combination       & Agenda & Chart  & AgendaA  \\
\hline
                  & k,r    &        & o \\
                  & r,     & k      & o \\
$\downarrow$(r,k) & kr     & r,k    & o \\
                  &        & r,k,kr & o \\
\hline
\end{tabular}}
\end{center}

All items containing an open substitution node are erased from the
chart (here, the tree anchored by \koweynatlang{runs}) as there is
no hope that they will be made complete in this phase.
The agenda
is then reinitialised to the content of the chart and the chart to the
content of the auxiliary agenda (here \koweynatlang{often}).
The \textbf{adjunction phase} proceeds much like the previous phase,
except that now all possible adjunctions are performed.  When
the agenda is empty once more, the items in the chart whose semantics
matches the input semantics are selected, and their strings printed
out, yielding in this case the sentence \koweynatlang{Kelvin runs often}.

\begin{center}
{\footnotesize
\begin{tabular}{|p{6em}|p{9em}|p{6em}|p{4em}|}
\hline
Combination   & Agenda & Chart  & Results  \\
\hline
              & k,kr   & o      &          \\
              & kr     & o      &          \\
$\star$(kr,o) & kro    & o      &          \\
              &        & o      & kro      \\
\hline
\end{tabular}}
\end{center}

Note that the chart never changes during the adjunction phase.  This
is because we do not store new items onto the chart.  As each tree is
removed from the agenda, either we notice that it is semantically
complete and save it as a result, or we perform all adjunction
operations involving (i) the tree and (ii) a tree from the chart, and
put any resulting derived trees back onto the agenda.  The tree itself
is no longer of any use and may be discarded.  We will see more
implications of this below, when we have more than one auxiliary tree
that can adjoin into the same node.

\subsubsection{Kelvin runs often in Colorado}
Now, a slightly more complicated example.  Here, we shall see auxiliary
tree with substitution nodes, as well as a lexical selection which leads
to more than one result.  Our new input semantics
is \koweysemexpr{l1:name(k,Kelvin), l0:often(r), l0:run(r,k), l0:in(r,i),
l0:place(i,Colorado)}, basically the same as before with two new literals.
The lexically selected trees (Figure \ref{fig:kroic}) are the same as
before plus trees for \koweynatlang{in} and \koweynatlang{Colorado}.
The trees for \koweynatlang{in}, \koweynatlang{Colorado},
\koweynatlang{Kelvin} and \koweynatlang{runs} are placed on the agenda,
whereas the tree for \koweynatlang{often} is placed on the auxiliary
agenda.

%\begin{figure}[htpb]
%\begin{center}
%\KelvinRunsOftenInColorado
%\end{center}
%\caption{Kelvin runs often in Colorado}
%\label{fig:kroic}
%\end{figure}

Again, the substitution phase explores all possible substitutions, of
which there are two: \koweynatlang{Kelvin} into \koweynatlang{runs} to
get \koweynatlang{Kelvin runs}, and \koweynatlang{Colorado} into
\koweynatlang{in} to get \koweynatlang{in Colorado}.  The case of
\koweynatlang{in Colorado} is particularly interesting because it is a
syntactically complete auxiliary tree
\koweytree{V(V*,PP(P(in),NP(Colorado)))}, in the sense that it has no open
substitution sites.  When we produce such trees, we transfer them to the
auxiliary agenda, so that they can be used during the adjunction phase.

\begin{center}
{\footnotesize
\begin{tabular}{|p{8em}|p{6em}|p{6em}|p{5em}|}
\hline
Combination        & Agenda     & Chart      & AgendaA  \\
                     \hline
                   & i,c,k,r    &            & o        \\
                   & c,k,r      & i          & o        \\
$\downarrow$(i,c)  & ic,k,r     & i,c        & o        \\
                   & k,r        & i,c        & o,ic     \\
                   & r          & i,c,k      & o,ic     \\
$\downarrow$(r,k)  & kr         & i,c,k,r    & o,ic     \\
                   &            & i,c,k,r,kr & o,ic     \\
\hline
\end{tabular}}
\end{center}

As before, all items containing an empty substitution node are erased
from the chart (here, the trees anchored by \koweynatlang{runs}
and by \koweynatlang{in}).  The agenda is then reinitialised to the
content of the chart and the chart to the content of the auxiliary
agenda (here \koweynatlang{often} and \koweynatlang{in Colorado}).
The \textbf{adjunction phase} proceeds, performing all possible
adjunctions.  When the agenda is empty
once more, the items in the
chart whose semantics matches the input semantics are selected, and
their strings printed out, yielding in this case the sentences
\koweynatlang{Kelvin runs often in Colorado} and
\koweynatlang{Kelvin runs in Colorado often}.

\begin{center}
{\footnotesize
\begin{tabular}{|p{8em}|p{6em}|p{6em}|p{5em}|}
\hline
Combination                   & Agenda     & Chart      & Results  \\
                                \hline
                              & c,k,kr     & o, ic      & \\
                              & k,kr       & o, ic      & \\
                              & kr         & o, ic      & \\
$\star$(kr,o), $\star$(kr,ic) & kro,kric   & o, ic      & \\
$\star$(kro,ic)               & kroic,kric & o, ic      & \\
                              & kric       & o, ic      & kroic \\
$\star$(kric,o)               & krico      & o, ic      & kroic \\
                              &            & o, ic      & kroic, krico \\
\hline
\end{tabular}}
\end{center}

Here, we get more than one result because trees can combine in different
ways (we could also get more than one result if we had an ambiguous
lexical selection, but this is not the case here).  Note, by the way, that
the combinations here are a result of embedded adjunctions and not
multiple adjunctions.  The fact that \geni supports the former but not
the latter is a technical detail, and affects the number of outputs it
produces (allowing for multiple adjunctions introduces more output since
we cannot forbid embedded adjunctions), as well as the derivation trees
for the output.


\chapter{Paraphrase selection}
\label{cha:paraphrase}

Paraphrase selection can largely be performed by constraining the
lexical selection.  We will extend the lexical selection process and the
input semantics to express such
constraints, but before going into the details, let us mock up a small
usage scenario.  Consider the input semantics below.  We will imagine a
hypothetical grammar which associates it with three paraphrases:
\enumsentence{\semexpr{l1:give(e,j,b,m), l2:john(j), l3:mary(m)}\\
John gave the book to Mary.\\
Mary was given the book by John.\\
The book was given to Mary by John.
}
To restrict the selection, we might specify that the literal
\semexpr{l1:give(e,j,b,m)} be realised by a verb in the passive voice.
We rewrite the input semantics as follows with the expected effects:
\enumsentence{\semexpr{l1:give(e,j,b,m)[PassiveForm], l2:john(j),
l3:mary(m)}\\
\sout{John gave the book to Mary.}\\
Mary was given the book by John.\\
The book was given to Mary by John.
}
Adding more properties to the input semantics simply narrows down the
resulting output even more.  We restrict the literal further still:
\enumsentence{\semexpr{l1:give(e,j,b,m)[PassiveForm,
CanonicalToObject], l2:john(j), l3:mary(m)}\\
\sout{John gave the book to Mary.}\\
\sout{Mary was given the book by John.}\\
The book was given to Mary by John.
}

These tokens with which we enrich the input semantics are called
\jargon{tree properties}.  We will now see exactly how they are
used (Section \ref{sec:enriched-stuff} and \ref{sec:extending-lexsel})
and where they come from in practice (Section
\ref{sec:tree-properties}).  Building on these ideas, we will also see
how the properties can be used to narrow down the semantics so that we
produce at most \emph{one} output (Section
\ref{sec:producing-at-most-one-output}).

\subsection{Enriched lexical items and input semantics}
\label{sec:enriched-stuff}

The extensions require that we introduce enriched versions of the
lexicon and the input semantics, both taking tree properties into
account.  The basic idea is that linguists use tree properties to
describe lexical items and the surface realiser uses them to filter
the lexical selection.

\begin{definition}[Tree property]
A \jargon{tree property} is an identifier.  Some examples of tree
property are \semexpr{PassiveForm} and
\semexpr{CanonicalToObject}.
\end{definition}

\begin{definition}[Enriched lexical item]
An \jargon{enriched lexical item} is a triple
$\tuple{T,S,LTP}$.  $T$ and $S$ are the elementary tree and lexical
semantics as described in Definition \ref{def:lexical-item} (Page
\pageref{def:lexical-item}).  $LTP$ is a set of tree properties.
\end{definition}

\begin{definition}[Enriched input semantics]
An \jargon{enriched input semantics} is a set of enriched literals
of the form $L[tp_1,\ldots,tp_n]$, where $L$ is a saturated \ellyou
literal and $tp_1,\ldots,tp_n$ is a possibly empty set of tree
properties.  As a notational convenience, we omit the square brackets
when the set of tree properties is empty.
\end{definition}

\begin{definition}[Plain input semantics]
The \jargon{plain input semantics} is the result of stripping
away all the tree properties from an enriched semantics.  Given an
enriched input semantics $ES$ we say that the plain input semantics is
the set of literals of the form $L_i$ where $L_i[tp_1,\ldots,tp_n] \in
ES$
\end{definition}

\subsection{Enriched lexical selection}
\label{sec:extending-lexsel}

Taking tree properties into account consists of filtering the lexical
items so that only those with the desired tree properties are retained.
Given an enriched semantics $ES$,  we instantiate the lexicon as usual
(Section \ref{sec:lexical-selection}) and return the set of enriched
lexical items such that for each item $\tuple{T,S,LTP}$:
\begin{itemize}
\item (as usual) its instantiated semantics $S$ is non-empty and subsumes the plain
      input semantics;
\item for every enriched literal $L[tp_1,\ldots,tp_n]$ in the enriched
      input semantics $ES$, if $L \in S$ then $\{tp_1,\ldots,tp_n\}
      \subseteq LTP$.
\end{itemize}

\subsection{Where tree properties come from (metagrammars)}
\label{sec:tree-properties}

The tree selection mechanism requires that every lexical item in the
grammar ``possess'', i.e.\ be associated with, a set of tree properties.
There are many ways of achieving this intermediate goal.  One possible
solution might be manual annotation, but this is neither desirable nor
necessary.  It is undesirable because realistic TAG grammars are large
enough to make such a process error-prone and cumbersome.  It is
unnecessary in the case of grammars built from the
XMG metagrammar compiler because the annotations are already
encoded in another resource, the metagrammar from which it was compiled.
See the XMG documentation for more details.

\subsection{Producing at most one output}
\label{sec:producing-at-most-one-output}

The more tree properties used to enrich the input semantics, the fewer
lexical items are selected.  Pushing this to its logical conclusion,
one could conceivably add enough tree properties for each literal to be
realised by at most one lexical item each.\footnote{We say at most
because of lexical items with a multi-literal semantics}  For the most
part having an ambiguity-free lexical selection is enough to guarantee
that the surface realiser returns a single paraphrase, the exceptions
being inputs where the lack of word order constraints (e.g., intersective
modifiers) comes into play.  The tree properties associated with each
lexical item must \emph{uniquely} identify that item.  In other words,
each enriched lexical item must be associated with a so-called tree
identifier:

\begin{definition}[tree identifier]
In an FB-LTAG grammar, a \jargon{tree identifier} is a set of tree
properties $I$.  If in a set of lexical entries, there is only
one enriched lexical item $\tuple{T,S,LTP}$ such that $LTP
\subseteq I$, we say that the tree identifier is unique to the set.
\end{definition}

What is important here is not just the fact that tree identifiers are
unique.  After all, the elementary trees produced by XMG already have
unique names like \verb!Tn0Vn1-387! which would technically allow one to
achieve the same result.  But these do not have the same practical use
as proper tree identifiers, the main reason being that they are
completely arbitrary, and are not imbued with the same linguistic
significance as tree properties.  The reason linguistic significance
matters is that it gets us closer to the objective of choosing
\emph{contextually appropriate} paraphrases.  Basically, we need some
means of representing linguistic alternatives and selecting from them.
We believe that disjunctions in the metagrammar serves as a mechanism
for representing the alternatives, and that tree properties provide the
mechanism for selecting among them.
